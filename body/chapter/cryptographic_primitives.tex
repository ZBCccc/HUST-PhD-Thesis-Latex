\chapter{预备知识}
\section{密文策略属性基加密}
密文策略属性基加密(CP-ABE)由Bethencourt等人首次正式提出 \cite {bethencourt2007cpabe},是一种能够对加密数据实现细粒度访问控制的密码学原语。与传统公钥密码体制不同 —— 传统公钥密码是面向特定目标接收者加密数据,而 CP-ABE 则是基于一组属性定义的访问策略对数据进行加密。

在 CP-ABE 方案中,用户的私钥与一组属性 $S$ 相关联(例如,\textit{“部门:计算机科学”},\textit{“角色:学生”}),而密文则与一个访问结构 $\mathbb{A}$ 相关联(例如,\textit{“部门:计算机科学”} AND (\textit{“角色:教师”} OR \textit{“角色:学生”}))。只有当用户的属性集 $S$ 满足访问结构 $\mathbb{A}$ 时,用户才能解密该密文。这种一对多的加密机制特别适用于需要灵活授权策略的安全云存储和可搜索加密方案。
    
\subsection{形式化定义}

形式上,一个 CP-ABE 方案 $\Pi$ 由四个概率多项式时间(PPT)算法组成:$\mathsf{Setup}$、$\mathsf{KeyGen}$、$\mathsf{Encrypt}$ 和 $\mathsf{Decrypt}$。

\begin{definition}[CP-ABE 方案]
一个 CP-ABE 方案由如下算法元组定义:

\begin{itemize}
    \item $\mathsf{Setup}(1^\lambda) \to (\mathsf{PK}, \mathsf{MK})$:  
    初始化算法以安全参数 $\lambda$ 作为输入,输出公共参数 $\mathsf{PK}$ 和主密钥 $\mathsf{MK}$。公共参数通常包括双线性群及其生成元的描述。

    \item $\mathsf{KeyGen}(\mathsf{MK}, S) \to \mathsf{SK}_S$:  
    密钥生成算法以主密钥 $\mathsf{MK}$ 和描述用户的一组属性 $S$ 作为输入,输出与属性集 $S$ 相关联的私钥 $\mathsf{SK}_S$。

    \item $\mathsf{Encrypt}(\mathsf{PK}, M, \mathbb{A}) \to \mathsf{CT}$:  
    加密算法以公共参数 $\mathsf{PK}$、消息 $M$ 和定义在属性全集上的访问结构 $\mathbb{A}$ 作为输入,输出密文 $\mathsf{CT}$,该密文隐式地包含访问结构 $\mathbb{A}$。

    \item $\mathsf{Decrypt}(\mathsf{PK}, \mathsf{CT}, \mathsf{SK}_S) \to M$:  
    解密算法以公共参数 $\mathsf{PK}$、包含访问结构 $\mathbb{A}$ 的密文 $\mathsf{CT}$,以及与属性集 $S$ 相关联的私钥 $\mathsf{SK}_S$ 作为输入。如果属性集 $S$ 满足访问结构 $\mathbb{A}$(记作 $S \models \mathbb{A}$),算法输出消息 $M$;否则输出失败符号 $\bot$。
\end{itemize}
\end{definition}

\subsection{访问结构}

访问结构在 CP-ABE 中通常通过单调访问树或线性秘密共享方案(LSSS)来实现。访问结构 $\mathbb{A}$ 是属性全集 $\mathcal{U}$ 的非空子集的集合。如果对于任意集合 $B \in \mathbb{A}$ 且 $B \subseteq C$,都有 $C \in \mathbb{A}$,则称 $\mathbb{A}$ 是\textbf{单调的}。

在实际构造中(例如 Waters~\cite{waters2011cpabe}),访问策略通常表示为 LSSS 矩阵 $(M, \rho)$,其中 $M$ 是一个 $\ell \times n$ 的矩阵,$\rho$ 是一个将矩阵行映射到属性的函数。
\subsection{安全模型}

CP-ABE 的标准安全性定义是 \textit{选择明文攻击下的不可区分性}(IND-CPA)。该定义通过一个挑战者 $\mathcal{C}$ 和一个对手 $\mathcal{A}$ 之间的安全游戏来描述:

\begin{enumerate}
    \item \textbf{初始化:} $\mathcal{C}$ 运行 $\mathsf{Setup}(1^\lambda)$ 并将 $\mathsf{PK}$ 发送给 $\mathcal{A}$。
    
    \item \textbf{第一阶段:} $\mathcal{A}$ 针对属性集 $S_1, \dots, S_{q_1}$ 进行密钥生成查询。
    
    \item \textbf{挑战:} $\mathcal{A}$ 提交两个等长消息 $M_0, M_1$ 和一个挑战访问结构 $\mathbb{A}^*$,约束条件是第一阶段中查询的属性集 $S_i$ 都不满足 $\mathbb{A}^*$(即 $\forall i, S_i \not\models \mathbb{A}^*$)。$\mathcal{C}$ 随机翻转一枚硬币 $b \in \{0, 1\}$ 并计算 $\mathsf{CT}^* \leftarrow \mathsf{Encrypt}(\mathsf{PK}, M_b, \mathbb{A}^*)$。$\mathcal{C}$ 将 $\mathsf{CT}^*$ 发送给 $\mathcal{A}$。
    
    \item \textbf{第二阶段:} $\mathcal{A}$ 可继续对集合 $S_{q_1+1}, \dots, S_q$ 进行密钥查询,同样要求查询的属性集都不满足 $\mathbb{A}^*$。
    
    \item \textbf{猜测:} $\mathcal{A}$ 输出猜测 $b'$。
\end{enumerate}

对手的优势定义如下:
\begin{equation}
    \mathbf{Adv}_{\mathcal{A}}^{\text{CP-ABE}}(\lambda) = \left| \Pr[b' = b] - \frac{1}{2} \right|
\end{equation}
如果该优势在 $\lambda$ 上可忽略,则称该方案满足 IND-CPA 安全性。