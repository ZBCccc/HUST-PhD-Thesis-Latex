\chapter{基于Merkle Hash Tree的布隆过滤器构造与优化方法}

\section{设计动机与问题分析}
在多用户多关键词可搜索加密(SSE)系统中,搜索过程往往可以拆分为两类核心动作:一类负责定位候选结果,另一类负责对候选结果进行进一步筛选与判定。本文所说的“资格检验”,对应的正是后者,即服务器面对某个查询请求时,需要判断某个候选对象是否满足检索条件与访问约束,并据此决定其是否应当被纳入最终返回集合。该环节直接决定了用户看到的搜索结果是否正确、是否完整,也决定了系统在恶意云服务器模型下能否建立可审计的可信链路。

多关键词检索中常见的做法是选取一个主关键词用于拉取候选集合,再对其余关键词逐一执行资格检验。资格检验的职责可以概括为两点:其一,保证语义正确性,确保返回对象满足多关键词条件(例如合取查询中每个关键词均匹配)。其二,保证访问合规性,确保多用户环境下的权限边界不被破坏(例如某用户仅能检索其被授权范围内的数据)。一旦该环节缺少强约束,云服务器就拥有了在“筛选与判定”这一步随意操控输出的空间,系统很难仅凭最终结果判断其是否严格执行了协议。

现有多用户多关键词方案中,Bloom Filter 常被用作资格检验的加速结构,主要原因在于其空间开销小、查询速度快,能够将集合成员性测试压缩为常数次哈希与位访问。然而,Bloom Filter 的结构特性决定了它更接近一种高效但弱约束的工程化判定工具,难以承担“可验证资格检验”的安全目标,具体局限主要体现在以下三方面。

第一,Bloom Filter 属于概率型数据结构,存在不可避免的假阳性。假阳性意味着服务器即使严格按流程执行资格检验,也可能把本不满足条件的对象判定为“通过”,从而造成结果集合被污染。对 SSE 而言,这类污染不仅带来额外通信与解密开销,还可能放大访问模式暴露面,因为用户为甄别假阳性需要处理更多返回项。随着关键词数量上升、过滤条件增多,假阳性累积效应会更明显。

第二,Bloom Filter 不提供可审计的证据。其查询输出只是一个布尔值,服务器给出“通过”或“不通过”时,用户无法获得与该判定绑定的可验证证明材料。只要服务器在实现层面略作偏离,就可以对外宣称“资格检验已执行且结果如此”,而用户缺少独立手段验证。换句话说,Bloom Filter 的判定无法形成可验证的因果链条,无法把“服务器确实按规则进行了成员性测试”这一事实转换为用户可检查的密码学证据。

第三,在恶意服务器威胁模型下,资格检验环节存在清晰的攻击面。服务器可能出于节省计算、降低 I/O、隐藏存储状态、或定向干预检索结果等目的,实施以下行为:
其一,跳过部分关键词的资格检验,仅用主关键词候选集合直接构造返回结果,造成错误项混入。
其二,选择性过滤本应通过的对象,造成结果不完整,用户难以区分是数据本身不匹配还是服务器刻意省略。
其三,在多用户场景中,通过操控资格检验逻辑破坏权限边界,使未授权用户获得不应出现的候选对象,或让授权用户遭遇无故缺失。
上述行为共同指向同一问题:资格检验缺少“可验证性”,服务器的判定无法被用户独立复核。

因此,本章创新点一的直接动机是把资格检验从“高效但不可证明的概率判定”提升为“可验证、可追责的确定性判定”,并在不显著破坏原有 SSE 检索效率优势的前提下,让用户能够对服务器在资格检验阶段的行为进行有效核验。围绕这一动机,本文对资格检验机制提出如下设计目标:
一,确定性与可验证性。资格检验结果需要绑定可验证证据,用户可在本地完成验证,避免对服务器诚实性的假设。
二,兼容多关键词流程。机制能够自然嵌入“候选集生成 加资格检验”的常见检索范式,支持合取等多关键词语义。
三,适配多用户环境。机制能够与权限控制或资格判定的对象模型兼容,使资格检验同时覆盖语义正确性与访问合规性。
四,低额外开销。引入验证能力后,存储、计算与通信开销保持在可接受范围,使方案仍具备工程可落地性。

基于上述问题与目标,本文选择 Merkle Hash Tree(MHT)作为资格检验结构的核心原因在于其天然具备“认证型数据结构”的性质:数据集合通过根哈希形成全局承诺,服务器在回答成员性问题时可以提供一条从叶子到根的认证路径,用户利用公开根值即可验证该判定是否与被承诺的数据集合一致。这样一来,资格检验从单纯输出布尔值演进为“判定加可验证证明”的组合,用户可以对每一次资格检验的通过或不通过建立密码学意义上的可核验依据。进一步地,若采用适当的树构造方式(例如固定域映射的稀疏 Merkle 树思路),资格检验还能同时覆盖成员性与非成员性证明需求,为服务器的“未通过”判定提供同样可验证的证据基础。

总结而言,本节指出了多用户多关键词 SSE 中资格检验环节的关键地位,并分析了 Bloom Filter 方案在可验证性、确定性与恶意服务器防护方面的结构性不足。本章后续内容将围绕 MHT 的结构设计、资格检验算法与安全性论证展开,给出一种能对资格检验结果提供确定性证明、并能抑制服务器在资格检验阶段作弊空间的实现路径。