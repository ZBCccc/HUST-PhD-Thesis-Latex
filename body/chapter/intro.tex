%%% mode: latex
%%% TeX-master: t
%%% End:

\chapter{绪论}
\section{研究背景与意义}
随着信息技术的飞速发展和数字化转型的深入推进,全球已全面步入大数据时代。根据国际数据公司(IDC)的统计,2020 年全球数据总量约为 64.2 ZB\cite{IDC2020DataAge}。随着云计算、物联网以及人工智能等技术的快速发展,全球数据规模持续高速增长。综合多家市场研究机构的估计结果,截至 2026 年前后,全球数据总量已增长至约 180 ZB的量级。云计算作为数字经济的关键基础设施,凭借其强大的计算能力、海量的存储空间以及按需付费的服务模式,彻底改变了传统的本地数据管理范式。越来越多的政府机构、企事业单位及个人用户选择将本地数据迁移至云端,以实现数据的弹性存储、资源共享与按需访问。

然而,数据外包存储在带来便利的同时,也引发了前所未有的安全与隐私挑战。外包数据往往包含高度敏感的信息,例如电子医疗系统中的个人健康记录、金融领域的交易流水、企业的核心知识产权以及政府的机密档案等。一旦这些数据在云端发生泄露或被非法滥用,将造成不可估量的经济损失和社会影响。近年来频繁发生的数据泄露事件,如2019年Capital One银行数据泄露案涉及1.06亿用户信息,以及2021年Facebook数据泄露事件影响5.33亿用户,进一步凸显了云环境下数据安全的紧迫性\cite{CNNBiz2019CapitalOneBreach, BusinessInsider2021FacebookBreach}。

在现有的云存储服务架构中,云服务提供商(Cloud Service Provider, CSP)通常被建模为"诚实但好奇"(honest-but-curious)的半可信实体。这意味着CSP虽然会忠实地执行用户发起的存储、检索等协议流程,不会主动篡改或破坏数据,但它们出于商业利益、法律强制或好奇心驱动,极有可能利用其掌握的底层基础设施权限,通过分析存储的数据副本、监控用户的查询请求频率以及挖掘数据的访问模式等手段,试图推断出数据所有者的隐私信息或商业机密\cite{yang2006privacy}。此外,外部高级持续性威胁(Advanced Persistent Threat, APT)攻击和内部恶意人员窃密等威胁因素,也使得云环境下的数据安全形势日益严峻\cite{cloud2011nist}。

为应对上述隐私威胁,同时满足《中华人民共和国数据安全法》、《个人信息保护法》以及欧盟《通用数据保护条例》(GDPR)等日益严格的数据合规性要求\cite{PRC2021DataSecurityLaw,GDPR2018EU},数据加密技术被广泛认为是云存储安全的最后一道防线。最直接的方法是在数据上传云端之前,使用AES、RSA等强加密算法对数据进行密文处理。然而,传统的加密技术在保护数据机密性的同时,彻底破坏了数据的自然语义和结构特征,导致云服务器无法在密文状态下执行关键词检索、排序、聚合等常规数据操作。在这种模式下,若用户需要查询包含特定关键词的文档,必须先将云端的海量密文数据全部下载至本地,解密后再进行检索。这种"下载-解密-检索"的处理模式不仅造成了巨大的网络带宽浪费和客户端计算资源消耗,在移动终端、物联网设备等计算能力受限的应用场景下更是完全不可行的。

如何在不泄露明文内容的前提下,实现对密文数据的高效检索,成为了学术界和工业界亟待解决的核心矛盾。可搜索加密(Searchable Encryption, SE)技术应运而生,它通过构造特殊的加密索引和搜索陷门(trapdoor),允许服务器在零知识的前提下完成关键词匹配,从而在保护数据隐私和保持数据可用性之间找到了理想的平衡点\cite{song2000practical}。

可搜索加密技术的研究可以追溯到2000年Song等人在IEEE S\&P会议上发表的开创性工作~\cite{song2000practical},该工作首次提出了在加密文档上进行关键词搜索的概念。随后,Goh在其博士论文中进一步完善了可搜索加密的理论基础~\cite{goh2003secure}。2004年,Boneh等人在EUROCRYPT会议上提出了公钥可搜索加密(Public-key Encryption with Keyword Search, PEKS)的概念~\cite{boneh2004public},使得第三方能够在不获知私钥的情况下进行搜索授权。2006年,Curtmola等人在ACM CCS会议上提出了适应性安全的对称可搜索加密方案,建立了更为严格的安全模型,并引入了前向隐私和后向隐私的概念~\cite{curtmola2006searchable}。

近年来,可搜索加密技术在多个维度取得了重要进展。在功能性方面,研究人员提出了支持范围查询、模糊关键词匹配、正则表达式搜索等高级检索功能的方案 \cite{li2010fuzzy,cao2013privacy}。在性能优化方面,基于倒排索引、布隆过滤器、乱序存储等技术的高效方案相继涌现 \cite{kamara2010cryptographic,abdalla2005searchable}。在安全性提升方面,针对文件注入攻击、查询恢复攻击等新型威胁模型,研究者们提出了具有更强隐私保护能力的防御机制 \cite{cash2015leakage,zhang2016all,damie2021highly}。

然而,现有的可搜索加密方案在面向实际应用部署时,仍然存在诸多局限性。首先,大多数方案仅支持单关键词搜索,无法满足用户复杂查询需求。其次,现有方案普遍假设单用户场景,缺乏对多用户权限管理和数据共享的有效支持。最后,现有方案缺乏对搜索结果完整性和正确性的可验证机制,存在恶意服务器返回不完整或错误搜索结果的安全风险 \cite{baek2008public}。

在实际的云存储应用场景中,数据共享与协作访问已成为常态。企业内部的项目文档需要在多个部门间共享,医疗机构的患者记录需要在不同科室间流转,学术研究机构的数据集需要向合作伙伴开放访问权限。这种多用户数据共享模式对可搜索加密技术提出了更高的要求。传统的单用户可搜索加密方案假设只有数据所有者能够生成搜索陷门,这在多用户环境下显然是不现实的。如何在保护数据隐私的同时,实现细粒度的访问控制和权限管理,成为亟待解决的关键问题。

与此同时,单关键词搜索的局限性在大数据时代愈发凸显。现实中的查询需求往往是复杂和多样化的,用户希望能够通过多个关键词的组合(如合取、析取查询)来精确定位所需信息,或者通过范围查询、排序检索等高级功能来获得更好的搜索体验。例如,在电子病历系统中,医生可能需要同时搜索"糖尿病"和"高血压"两个关键词来查找具有特定并发症的患者记录;在企业文档管理系统中,员工可能需要搜索包含"年度报告"或"财务分析"等关键词的文档。现有的单关键词可搜索加密方案无法直接支持这类复杂查询,通常需要进行多轮交互或客户端后处理,严重影响了搜索效率和用户体验。

多关键词可搜索加密在技术实现上面临诸多挑战。首先是索引构造的复杂性,需要设计能够同时支持多个关键词匹配的加密索引结构,同时保持较低的存储开销和检索复杂度。其次是搜索陷门的生成机制,需要考虑不同关键词之间的逻辑关系(合取、析取等)以及查询的可扩展性。最后是隐私保护的增强,多关键词搜索可能泄露更多的访问模式信息,需要设计更为精密的隐私保护机制。

搜索结果的完整性和正确性验证是可搜索加密系统面临的另一个关键挑战。在传统的明文搜索环境中,用户能够直接验证搜索结果的正确性,但在密文搜索环境下,这一问题变得更为复杂。半可信的云服务器可能出于降低计算成本、存储空间不足或恶意目的,有选择地返回部分搜索结果,或者返回与查询条件不匹配的错误结果。这种搜索结果的不完整性或错误性可能导致严重的业务后果,例如医疗诊断中遗漏关键病历信息、金融审计中丢失重要交易记录等。

可验证计算(Verifiable Computation)理论为解决这一问题提供了重要的理论基础。通过将可验证计算技术与可搜索加密相结合,可以构造出支持搜索结果完整性和正确性验证的可验证可搜索加密方案。这类方案不仅要保证搜索过程的隐私性,还要提供高效的验证机制,使得用户能够以较小的计算开销验证服务器返回的搜索结果是否完整且正确。

当前的可验证可搜索加密方案主要基于哈希链、默克尔树、累加器等密码学工具来构造验证信息。然而,现有方案在支持复杂查询(如多关键词搜索)时,验证信息的构造和维护变得异常复杂,验证开销也随之显著增加。如何在保持高效性的同时,为多用户多关键词可搜索加密提供可验证性保障,仍然是一个充满挑战的开放性问题。

综上所述,面向多用户的可验证多关键字可搜索加密方案研究具有重要的理论意义和实用价值。从理论层面,该研究将推动可搜索加密、访问控制、可验证计算等多个密码学分支的交叉融合,丰富和完善相关领域的理论体系。从实践层面,该研究将为云存储服务、大数据处理平台、物联网数据管理等实际应用提供更为安全、高效和可靠的隐私保护解决方案,对推动数字经济的健康发展具有重要意义。
\section{国内外研究现状}



\section{研究目标与内容}


\subsection{研究内容}


\subsection{研究创新点}

% \clearpage
\begin{figure}[!t]
    \centering
    \includegraphics[width=0.8\linewidth]{float/ch.intro/thesis_arch.png}
    \caption{\label{fig:body}本研究技术路线}
\end{figure}
\section{本文结构安排}
\subsection{技术路线\label{sec:thesisRoad}}
本研究以时间序列预测问题为背景,结合随机映射方法的建模效率与深度神经网络的学习潜能,基于SDNN预测建模技术进行方法研究和现实应用。
本研究的技术路线如\autoref{fig:body}所示,通过文献研究法分析SDNN预测建模技术的背景、现状与不足,归纳以卷积神经网络和循环神经网络结构为代表的既有SDNN建模方法,从参数优化与特征选择两方面,系统分析其在时间序列预测建模时的模型选择关键优化问题。

首先,针对不同特定代表结构下的SDNN模型参数优化个性问题展开研究,分别构造基于卷积结构的SDNN预测模型构造与优化方法,以及基于循环结构的SDNN预测模型构造与优化方法;而后,将典型且个性的特定结构研究在神经网络结构层面予以推广,关注以卷积结构与循环结构为代表的一般结构下SDNN预测模型的特征选择问题,提出兼容不同深度神经网络结构的二重特征结构选择方法,并进一步考虑混合不同神经网络结构的SDNN模型参数优化问题,提出SDNN模型的混合结构生长与优化方法。
在此技术路线中,结合主流人工合成时间序列数据与多种复杂的现实时间序列数据,通过准确性比较实验、收敛性验证实验和消融实验,结合定量对比与机器学习理论分析,验证所提方法的有效性。

\begin{figure}[!t]
    \centering
    \includegraphics[width=0.95\linewidth]{float/ch.intro/thesis_content.png}
    \caption{\label{fig:body.arch}本论文主要结构}
\end{figure}

% \clearpage
\subsection{论文结构}
全文共七章,主要结构如\cref{fig:body.arch}所示,各章内容如下:

第1章~绪论:介绍了本研究的背景与意义、所面临的关键问题、主要的研究内容与创新点、研究内容间的技术路线及全文结构。

第2章~国内外相关研究综述:介绍了时间序列预测问题的定义、机器学习预测建模技术至深度学习预测建模技术的演进与现状、随机映射方法的动机及随机映射深度学习预测建模技术的现状及不足。

第3章~基于卷积结构的SDNN预测模型构造与优化方法:提出误差反馈随机建模构造方法及贪心搜索参数优化方法,建立出一种高效且收敛的卷积结构SDNN预测模型,结合人工合成数据、流感阳率数据、原油价格数据和金融指数数据的预测问题进行实验验证。

第4章~基于循环结构的SDNN预测模型构造与优化方法:考量多种随机映射循环输出结构及提出基于粒子群优化的状态遮掩优化方法,建立出一种改进的循环结构SDNN预测模型,结合人工合成数据、电力负荷数据和室外温度数据的预测问题进行实验验证。

第5章~SDNN预测模型的二重特征结构选择方法:构造兼容卷积结构与循环结构深度神经网络的时间序列数据二维输入特征结构及提出基于树状帕尔森估计的二重特征结构选择方法,建立出基于改进特征选择的SDNN预测模型,结合人工合成数据、电力负荷数据和电力价格数据的预测问题进行实验验证。

第6章~SDNN预测模型的混合结构生长与优化方法:改进误差反馈生长策略及提出子网络参数的三阶段优化方法,建立出一种准确、收敛且稳健的混合结构SDNN预测模型,结合人工合成数据、空气污染数据和电力负荷数据的预测问题进行实验验证。

第7章~总结与展望:总结本研究的主要内容、不足之处及改进方向。

其中,第3章至第6章为本文的主要研究内容章节,其逻辑关系在于以下方面:

(1)第3章与第4章为并列关系,该两章主要研究内容是第5章与第6章研究内容的基础。具体地,本研究聚焦基于随机映射的时间序列深度学习预测建模技术,卷积结构和循环结构作为既有深度神经网络的代表结构,本文第3章与第4章分别针对深度神经网络基于随机映射方法在特定于卷积结构和循环结构实现下的构造与优化问题展开研究,解决了既有卷积结构实现方法的性能瓶颈,改进了既有循环结构实现方法的输入输出映射,提升了在卷积结构和循环结构这类特定深度神经网络结构实现上的SDNN预测模型准确性。

(2)基于输入特征选择视角,第5章是对第3、4章内容的改进。具体地,凝练第3章与第4章所述SDNN卷积结构和循环结构对时步多维度输入能力的支持共性,设计了兼容卷积结构和循环结构的SDNN预测模型二重特征结构选择方法,突破了既有SDNN预测模型特征选择方法在一维输入结构上的局限,提升了基于卷积结构和循环结构的一般深度神经网络结构下的SDNN预测模型准确性。

(3)基于整体参数优化视角,第6章是对第3、4章内容的改进。具体地,将第3章与第4章所述SDNN卷积结构和循环结构作为超参数控制下的子网络结构状态,提出自适应混合多种不同特定类型子网络的SDNN预测生长策略,建立对子网络在生长过程中超参数、编码过程权重参数和生成过程权重参数的整体参数优化方法,进一步提升了自适应选择卷积结构和循环结构等混合深度神经子网络结构下的SDNN预测模型准确性。



% 第六章 总结与展望:本章对全文的研究工作进行总结,指出本文研究的不足和未来的展望。

% % \section{课题来源}
% % 本课题来源于:

% % 国家自然科学基金面上项目:基于多输出支持向量回归的预测技术研究(立项时间:2016-01,项目编号:71571080)。

% % 国家自然科学基金面上项目:大数据环境下基于计算智能的预测建模技术及其在电力负荷预测中的应用(立项时间:2019-01,项目编号:71871101)。

% % 国家电网公司华中分部科技项目:华中区域共享型电力交易与服务平台关键技术研究(立项时间:2020-03,项目编号:52140019000U)。