%%% mode: latex
%%% TeX-master: t
%%% End:

\chapter{绪论}
\section{研究背景与意义}
随着云计算、物联网与大模型应用等技术的普及,数据规模正加速迈向更高量级。云计算作为数字经济的重要基础设施,依托弹性算力与海量存储资源,提供了低门槛、可扩展的按需服务模式,推动政府机构、企事业单位与个人用户将数据与计算任务外包至云端\cite{armbrust2010view}。

数据外包在提升资源利用率与协同效率的同时,也带来了更为严峻的安全与隐私风险。外包数据往往包含高度敏感的信息,例如电子医疗系统中的个人健康记录、金融交易流水、企业核心知识资产以及政府档案等。一旦发生泄露、滥用或越权访问,将造成显著的经济损失与社会影响\cite{ren2012security}。近年来多起数据泄露事件持续强化了这一风险的现实性,例如 2019 年 Capital One 事件与 2021 年 Facebook 事件涉及海量用户数据\cite{CNNBiz2019CapitalOneBreach, BusinessInsider2021FacebookBreach}。与此同时,合规要求也在不断加强,《中华人民共和国数据安全法》《个人信息保护法》以及欧盟《通用数据保护条例》(GDPR)等法规进一步推动云环境下的数据保护进入强约束阶段\cite{PRC2021DataSecurityLaw,GDPR2018EU}。

在典型云存储架构中,云服务提供商(Cloud Service Provider, CSP)常被建模为半可信实体\cite{rivest1978data}。该模型通常假设 CSP 会按协议执行存储与检索流程,但可能利用其对基础设施的控制能力,通过观察数据副本、查询频率与访问模式等侧信道信息推断隐私内容\cite{yang2006privacy}。除“好奇”行为外,云环境还面临外部高级持续性威胁(Advanced Persistent Threat, APT)与内部越权操作等风险,使得机密性保护与服务可信性保障同时成为系统设计的关键目标\cite{cloud2011nist}。

加密被广泛视为实现云端数据机密性保护的基础手段。数据在上传前进行加密可以降低泄露风险,但传统加密会将明文语义映射为随机化密文,服务器难以在密文状态下直接完成关键词检索、过滤与统计等操作。若用户需检索特定关键词相关文档,朴素方案往往要求客户端下载大量密文并在本地解密检索,通信与计算开销随数据规模增长而迅速上升,难以满足大规模数据与资源受限终端的应用需求。

为兼顾机密性与可用性,可搜索加密(Searchable Encryption, SE)通过构造加密索引与搜索陷门,使服务器能够在不获知明文内容的情况下完成匹配检索,从而在隐私保护与检索效率之间建立可行的折中\cite{song2000practical}。根据密钥体系的不同,SE 可分为公钥可搜索加密与对称可搜索加密,其中对称可搜索加密(Searchable Symmetric Encryption, SSE)因其计算与存储开销较低,成为面向云存储落地应用的研究重点\cite{curtmola2006searchable}。

然而,面向真实云服务部署时,现有 SSE 方案仍存在两类突出问题。其一,多用户共享与协作访问已成为常态,系统需要支持“数据拥有者将加密数据共享给多个授权用户”的使用模式,并在用户加入、撤销与权限变更等操作下保持安全与可管理性\cite{fu2015achieving}。其二,现实查询往往具有多关键词与布尔组合特征,单关键词检索难以满足精细化的信息获取需求\cite{sun2013privacy,cao2013privacy}。更重要的是,很多 SSE 工作主要在泄露受限的隐私模型下讨论机密性保证\cite{bosch2014survey},而对云服务器可能偏离协议执行、返回不完整或错误搜索结果的问题关注不足\cite{chai2012vsse}。该类偏离未必破坏密文机密性,却会直接影响检索服务的正确性与可用性。在医疗、审计、应急指挥等高依赖检索结果的业务场景中,结果缺失或错误可能带来严重后果。

因此,如何在多用户多关键词 SSE 场景下,在保持检索效率与隐私保护特性的同时,为搜索过程提供可验证性保障,使授权用户能够有效检测服务器返回结果的完整性与正确性,具有重要的理论意义与工程价值。从理论层面,这一问题涉及可搜索加密、访问控制与可验证计算等方向的交叉融合;从应用层面,该研究有助于提升云端密文检索服务的可信性,为数据要素流通与合规治理提供支撑。

\section{国内外研究现状}
可搜索加密围绕“密文可检索”这一核心目标,经历了从可行性验证到形式化安全建模、从静态数据集到动态更新、从单用户到多用户共享、从单关键词到多关键词组合查询的持续演进。以下从 SSE 安全性演进、多用户 SSE、以及可验证可搜索加密三个方面概述相关研究进展,并归纳面向本文问题设置的关键挑战。

\subsection{可搜索加密及其安全性演进}
Song 等人最早提出具有实用意义的对称可搜索加密方案,证明了在加密文档上执行关键词搜索的可行性\cite{song2000practical}。随后,Goh 通过安全索引结构进一步推动了可搜索加密的工程化探索\cite{goh2003secure}。在公钥体系下,Boneh 等人提出 PEKS 概念,为第三方授权搜索提供了新的设计空间\cite{boneh2004public}。在对称体系下,Curtmola 等人给出了更严格的非自适应与自适应安全定义,并构造可证明安全的 SSE 方案,奠定了后续形式化分析的基础\cite{curtmola2006searchable}。

随着云存储对数据动态更新需求的增强,动态 SSE 成为研究重点之一。动态场景引入了更新模式等额外泄露面,使得隐私保护面临新的挑战。为此,研究者提出前向隐私与后向隐私等安全属性,并通过索引状态演化、受控随机化与结构化更新等机制降低历史查询与新增或删除文档之间的关联风险。另一方面,学术界也逐渐认识到泄露受限安全并不等同于对推断攻击的全面抵抗,文件注入攻击、查询恢复攻击等工作表明,结合背景知识与统计特征,攻击者可能从访问模式与结果规模中推断查询内容\cite{cash2015leakage,zhang2016all,damie2021highly}。

在系统构造层面,许多高效 SSE 方案以倒排索引为核心,结合布隆过滤器、乱序存储与结构化加密框架,构造可扩展的密文索引,并使查询复杂度主要与匹配文档数量相关\cite{kamara2010cryptographic,abdalla2005searchable}。这类结构显著提升了云端检索的吞吐能力,也为多关键词查询提供了可复用的工程基础,但索引结构的复杂化会放大“结果可验证”的实现难度,服务器可操纵的返回路径增多后,用户更难在密文域下直接判断结果是否被删减或被替换。

在查询表达能力方面,早期 SSE 主要支持单关键词检索。为了满足精细化检索需求,研究者提出支持布尔查询、模糊匹配与更复杂检索类型的方案\cite{li2010fuzzy,cao2013privacy}。多关键词查询通常需要同时兼顾表达能力与泄露控制,尤其是在合取查询或析取查询中,结果集合间的关系可能被服务器观察,从而放大查询语义泄露风险。因此,多关键词 SSE 往往在索引结构、交互轮次与系统开销之间形成新的权衡。

总体而言,SSE 在机密性建模、动态更新与多关键词检索方面已形成较为丰富的研究脉络,但多数工作仍以“服务器遵循协议”作为正确性前提,缺乏对结果完整性与正确性的显式验证机制。

\subsection{多用户可搜索加密研究现状}

多用户场景通常对应“一个数据拥有者对多个数据用户授权搜索与访问”的体系结构。与单用户模型相比,多用户模型需要处理密钥与权限分发、用户撤销、跨用户串通等问题,且这些机制需与 SSE 的索引与陷门生成过程协同设计,否则会引入额外泄露或显著增加系统开销。

早期的直接扩展方式通常采用共享搜索密钥,使所有授权用户都能生成搜索陷门。该方式在可管理性与安全性方面存在明显局限,单个用户密钥泄露即可导致系统级风险,撤销与权限变更也会引发高成本的密钥重分发。为降低数据拥有者的在线负担并提升授权粒度,多客户端 SSE 路线引入授权令牌、能力限制或辅助授权信息,使数据拥有者能够在较低交互成本下授权用户发起受限查询,并支持一定程度的动态更新\cite{fu2015achieving}。在更强授权语义下,属性基加密与关键词搜索结合的路线通过将访问策略与属性绑定,实现细粒度授权与抗串通能力,但往往带来更高的计算代价与更复杂的撤销机制。另有研究利用代理重加密等原语,将密钥转换与权限迁移外包给半可信代理,以支持用户加入与撤销并降低重加密成本,但需要谨慎界定代理能力边界并处理一致性与泄露问题。

综合来看,多用户可搜索加密已经在授权机制与工程可用性方面取得进展,但在多关键词组合查询与可验证性要求同时存在的情况下,索引维护、授权一致性与验证信息更新将面临更高复杂度。

\subsection{可验证可搜索加密技术研究现状}
在云环境中,服务器出于资源节省、故障或恶意目的可能返回不完整甚至错误的搜索结果。为提升服务可信性,可验证可搜索加密(Verifiable Searchable Encryption, VSE)通过引入认证数据结构或可验证计算技术,使用户在不获取明文索引的前提下验证结果的完整性与正确性\cite{baek2008public}。

现有 VSE 方案常借助哈希链、默克尔哈希树、累加器等工具,为倒排索引或搜索路径构造认证信息,使服务器在返回检索结果的同时给出可验证证明。该思路在单关键词或简单查询条件下较易实现,但在多关键词组合查询中,验证对象往往从“单个倒排列表”扩展为多个列表之间的组合关系,使得证明结构与更新维护成本显著上升。进一步地,在多用户场景下,授权机制会引入额外的资格检验或访问控制流程,验证机制需要与授权一致性相匹配,否则可能出现“授权通过但证明不可验证”或“证明可验证但权限越界”等问题。现有研究在多用户、多关键词与动态更新同时成立的综合设置下仍缺乏兼顾效率与可验证性的统一框架。

综上,面向本文关注的多用户多关键词 SSE 场景,仍存在三方面挑战。首先,多关键词组合检索带来索引与泄露的耦合增长,验证信息的规模控制与更新维护成为瓶颈。其次,多用户授权引入资格检验与撤销管理,系统需要保证授权结果与验证结果的一致性,并防止服务器在授权阶段进行选择性作弊。最后,在保持原有 SSE 检索复杂度优势的前提下引入可验证机制,需要避免额外索引结构导致的存储与通信成本失控,这对可部署性提出了更高要求。

\section{研究目标与内容}
本文面向多用户多关键词对称可搜索加密场景,研究在泄露受限隐私保护模型下引入轻量可验证机制的问题。目标是在保持原有检索效率与动态更新能力的基础上,使授权用户能够验证云服务器在搜索流程中的关键步骤是否按协议执行,并检测搜索结果的完整性与正确性,从而提升密文检索服务的可信性。

围绕上述目标,本文首先给出多用户多关键词 SSE 的系统模型与威胁模型,明确云服务器可能偏离协议的攻击面与验证需求。在此基础上,本文将可验证性需求拆分为两个与工程实现紧密相关的环节:资格检验阶段与搜索结果返回阶段。前者关注服务器是否正确执行授权过滤与成员性判断,后者关注服务器是否返回了与查询条件一致且未被删减的结果集合。

针对资格检验阶段,本文研究如何在现有基于概率数据结构的设计上引入确定性证明,使用户能够验证服务器的过滤判断。针对搜索结果返回阶段,本文研究如何利用更新阶段的可信计算,将“正确结果的校验信息”与既有索引结构融合,在不显著增加索引规模的条件下支持结果正确性验证。最后,本文对提出机制进行安全性与可验证性分析,并通过实验评估验证机制引入后的系统开销与性能影响。

本文的创新点主要体现在两个方面。

第一,提出基于默克尔哈希树的可验证资格检验机制。针对多用户场景中常见的资格检验流程,本文以认证数据结构替换概率性成员判断结构,为服务器的资格检验结果提供确定性的成员性证明与可验证路径,使授权用户能够检测服务器在资格检验阶段的选择性遗漏与伪造行为,从结构层面增强系统在多用户环境下的可验证性。

第二,提出基于嵌入式承诺的搜索结果正确性验证机制。针对服务器在检索阶段可能伪造地址或删减结果的问题,本文在数据更新阶段引入承诺思想,将与“正确搜索结果集合”相关的校验信息嵌入既有索引结构中,使用户在搜索阶段能够对服务器返回的结果集合进行一致性验证。在该机制中,验证信息与索引条目同构维护,无需额外引入独立验证索引,从而在可验证性与系统开销之间取得更适合部署的平衡。

\section{论文结构}
本文共分为五章,各章内容安排如下。

第一章为绪论,阐述研究背景与意义,梳理国内外研究现状并归纳关键挑战,概述本文研究目标、研究内容与创新点,并给出全文结构安排。

第二章介绍可搜索加密基础方案与相关密码学预备知识。首先给出系统模型与威胁模型,其次介绍多用户多关键词 SSE 的典型基础方案与搜索流程,并分析现有资格检验结构及其局限,最后说明后续章节所需的密码学原语与技术。

第三章提出基于默克尔哈希树的可验证资格检验机制。该章围绕资格检验阶段的攻击面与验证需求给出结构设计与算法流程,并对可验证性与安全性进行分析。

第四章提出基于嵌入式承诺的搜索结果正确性验证机制。该章给出服务器伪造或删减结果的攻击模型,描述承诺嵌入与搜索验证流程,并对结果正确性与完整性进行安全性分析。

第五章给出实验评估与总结展望。该章通过实验分析可验证机制引入后的性能与开销变化,验证方案的可行性,并总结全文工作与未来研究方向。
