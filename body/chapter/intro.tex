%%% mode: latex
%%% TeX-master: t
%%% End:

\chapter{绪论}
\section{研究背景与意义}
随着云计算、物联网与大模型应用等技术的普及,数据规模正加速迈向更高量级。云计算作为数字经济的核心基础设施,依托弹性算力与海量存储资源,提供了低门槛、可扩展的按需服务模式,推动政府机构、企事业单位与个人用户将数据与计算任务外包至云端\cite{armbrust2010view}。然而,数据外包在提升资源利用率与协同效率的同时,也带来了更为严峻的安全与隐私风险。外包数据往往包含高度敏感的信息,例如电子医疗系统中的个人健康记录、金融交易流水、企业核心知识资产以及政府档案等。一旦发生泄露、滥用或越权访问,将造成显著的经济损失与社会影响\cite{ren2012security}。近年来,Capital One 数据泄露事件(2019)与 Facebook 用户数据泄露事件(2021年)均涉及海量敏感信息外泄\cite{CNNBiz2019CapitalOneBreach, BusinessInsider2021FacebookBreach},持续印证了这一风险的现实性与紧迫性。与此同时,《中华人民共和国数据安全法》《个人信息保护法》以及欧盟《通用数据保护条例》(GDPR)等法规的相继实施\cite{PRC2021DataSecurityLaw,GDPR2018EU},进一步推动云环境下的数据保护进入强合规约束阶段。

在典型云存储架构中,云服务器常被建模为"半诚实"(semi-honest)实体\cite{rivest1978data}。该安全模型假设云服务器会按协议规范执行存储与检索流程,但可能利用其对基础设施的控制能力,通过观察数据副本、查询频率与访问模式等侧信道信息推断用户隐私内容\cite{yang2006privacy}。除半诚实行为外,云环境还面临外部高级持续性威胁(Advanced Persistent Threat, APT)与内部越权操作等风险,使得机密性保护与服务可信性保障同时成为系统设计的核心目标\cite{cloud2011nist}。

加密技术被广泛视为实现云端数据机密性保护的基础手段。数据在上传前进行加密可有效降低静态数据泄露风险,但传统加密方案将明文语义映射为随机化密文,导致云服务器难以在密文状态下直接完成关键词检索、过滤与统计等操作。若用户需检索包含特定关键词的文档,朴素方案往往要求客户端下载全部相关密文并在本地解密后检索,其通信与计算开销随数据规模增长而迅速上升,难以满足大规模数据存储与资源受限终端的实际应用需求。

为在机密性与检索可用性之间建立有效折中,可搜索加密(Searchable Encryption, SE)技术通过构造加密索引与搜索陷门(trapdoor),使云服务器能够在不获知明文内容的情况下完成匹配检索\cite{song2000practical}。根据密钥体系的不同,可搜索加密可分为公钥可搜索加密(Public-key Encryption with Keyword Search, PEKS)\cite{boneh2004public} 与对称可搜索加密(Searchable Symmetric Encryption, SSE)\cite{curtmola2006searchable}两大类。其中,SSE 因其计算与存储开销显著低于公钥体系方案,成为面向云存储实际部署的研究重点。。

然而,面向真实云服务部署场景时,现有 SSE 方案仍存在两类突出问题。
其一,多用户协作访问需求与授权机制复杂性之间的矛盾。在企业文档管理、医疗信息共享、多部门协同办公等典型应用场景中,数据拥有者需要将加密数据授权给多个用户进行检索与访问\cite{fu2015achieving}。多用户场景引入了密钥分发、用户撤销、权限动态变更与跨用户串通防护等新挑战,且这些机制需与 SSE 的索引结构、陷门生成过程协同设计,否则将引入额外泄露或显著增加系统开销。
其二,多关键词组合查询的表达能力需求与隐私泄露控制之间的张力。现实检索场景中的查询往往具有多关键词与布尔组合特征,单关键词检索难以满足精细化的信息获取需求\cite{sun2013privacy,cao2013privacy}。多关键词查询需要在表达能力与泄露控制之间寻求平衡,尤其是在合取或析取查询中,结果集合之间的关系可能被云服务器观察,从而放大查询语义泄露风险。
更为关键的是,现有多数 SSE 研究主要在泄露受限的隐私模型下讨论机密性保证\cite{bosch2014survey},而对云服务器可能偏离协议执行、返回不完整或错误搜索结果的问题关注不足\cite{chai2012vsse}。该类协议偏离行为虽未必直接破坏密文机密性,却会严重影响检索服务的正确性与可用性。在医疗诊断、审计追溯、应急指挥等高度依赖检索结果完整性的业务场景中,结果缺失或错误可能导致严重后果。因此,如何使授权用户能够有效检测云服务器返回结果的完整性与正确性,即搜索结果的可验证性问题,具有重要的研究价值。

本文研究多用户多关键词 SSE 场景下的可验证性问题,其研究意义体现在理论层面与应用层面两个维度。
\begin{enumerate}
    \item 理论意义:可验证可搜索加密(Verifiable Searchable Encryption, VSE)涉及可搜索加密、访问控制与可验证计算等多个密码学分支的交叉融合。在多用户多关键词设置下,验证机制需要同时处理授权资格检验的正确性与搜索结果集合的完整性,这对认证数据结构的设计与更新维护策略提出了新的理论挑战。本文探索在该复合场景下引入轻量可验证机制的技术路径,有助于拓展可验证计算理论在密文检索领域的应用边界。
    \item 应用价值:在云存储服务日益普及的背景下,密文检索服务的可信性直接影响用户对云平台的信任度与采纳意愿。通过为搜索过程提供可验证性保障,用户能够在不完全信任云服务器的前提下获得对检索结果正确性的确信,这对于医疗健康、金融审计、政务档案等对数据完整性要求严格的应用领域尤为重要。本文研究有助于提升云端密文检索服务的可信性,为数据要素流通与合规治理提供技术支撑。
\end{enumerate}



\section{国内外研究现状}
可搜索加密围绕"密文可检索"这一核心目标,经历了从可行性验证到形式化安全建模、从静态数据集到动态更新、从单用户到多用户共享、从单关键词到多关键词组合查询的持续演进。本节从 SSE 基础理论与安全性演进、多用户 SSE、以及可验证可搜索加密三个方面梳理相关研究进展。这三个方面分别对应本文研究问题的三个核心维度:SSE 基础理论为本文提供检索机制基础,多用户 SSE 为授权框架设计提供参考,可验证 SSE 则是本文技术贡献的直接相关领域。在各部分综述基础上,本节末尾将归纳面向本文问题设置的关键技术挑战。

\subsection{对称可搜索加密基础理论与安全性演进}
\textbf{奠基性工作与形式化安全模型}。Song 等人\cite{song2000practical}于 2000 年首次提出具有实用意义的对称可搜索加密方案,通过在加密文档中嵌入可搜索结构,证明了在密文上执行关键词搜索的可行性。该工作开创了 SSE 研究领域,但其方案的搜索复杂度与文档总长度线性相关,且缺乏形式化的安全性定义。随后,Goh \cite{goh2003secure} 引入安全索引(secure index)概念,利用布隆过滤器为每个文档构建关键词索引,将搜索复杂度降至与文档数量线性相关,并给出了基于模拟的安全性定义,推动了 SSE 的工程化探索。在公钥体系下,Boneh 等人\cite{boneh2004public}提出公钥可搜索加密(PEKS)概念,支持第三方在不持有私钥的情况下生成搜索陷门,为多方授权搜索提供了新的设计空间,但其计算开销显著高于对称体系方案。
Curtmola 等人\cite{curtmola2006searchable} 的工作是 SSE 形式化安全理论的里程碑。该文给出了非自适应安全(non-adaptive security)与自适应安全(adaptive security)两种安全性定义,前者假设敌手在获取加密索引前提交所有查询,后者允许敌手根据已观察到的信息自适应地选择后续查询。基于该形式化框架,作者构造了可证明安全的 SSE 方案,其搜索复杂度仅与匹配文档数量相关(即次线性搜索),奠定了后续 SSE 安全性分析的理论基础。

\textbf{动态 SSE 与前向/后向隐私}。早期 SSE 方案主要针对静态数据集设计。随着云存储对数据动态更新需求的增强,支持文档添加与删除的动态 SSE(Dynamic SSE)成为研究重点。动态场景引入了更新模式(update pattern)等额外泄露面,使得隐私保护面临新的挑战。具体而言,若敌手能够关联更新操作与后续查询结果,则可能推断新增文档的关键词信息或已删除文档的历史存在性。为应对上述泄露风险,研究者提出了前向隐私(forward privacy)与后向隐私(backward privacy)等安全属性。前向隐私要求新增文档的更新操作不泄露其与历史查询的关联性,后向隐私要求删除操作后,后续查询结果不泄露已删除文档的历史信息。Bost\cite{bost2016ovarphiovarsigma}提出的Σoϕoς方案是前向安全 SSE 的代表性工作,通过陷门置换与状态演化机制实现高效的前向隐私保护。在实现后向隐私方面,Bost 等人\cite{bost2017forwardbackward}的工作具有奠基性意义,该文首次提出了同时支持前向隐私与后向隐私的动态 SSE 构造,通过约束伪随机函数(Constrained PRF)与可穿透加密(Puncturable Encryption)等密码学原语,严格保证了删除操作后的历史信息不泄露,成为后续研究的重要基础。


\subsection{多用户可搜索加密研究现状}

多用户场景通常对应“一个数据拥有者对多个数据用户授权搜索与访问”的体系结构。与单用户模型相比,多用户模型需要处理密钥与权限分发、用户撤销、跨用户串通等问题,且这些机制需与 SSE 的索引与陷门生成过程协同设计,否则会引入额外泄露或显著增加系统开销。

早期的直接扩展方式通常采用共享搜索密钥,使所有授权用户都能生成搜索陷门。该方式在可管理性与安全性方面存在明显局限,单个用户密钥泄露即可导致系统级风险,撤销与权限变更也会引发高成本的密钥重分发。为降低数据拥有者的在线负担并提升授权粒度,多客户端 SSE 路线引入授权令牌、能力限制或辅助授权信息,使数据拥有者能够在较低交互成本下授权用户发起受限查询,并支持一定程度的动态更新\cite{fu2015achieving}。在更强授权语义下,属性基加密与关键词搜索结合的路线通过将访问策略与属性绑定,实现细粒度授权与抗串通能力,但往往带来更高的计算代价与更复杂的撤销机制。另有研究利用代理重加密等原语,将密钥转换与权限迁移外包给半可信代理,以支持用户加入与撤销并降低重加密成本,但需要谨慎界定代理能力边界并处理一致性与泄露问题。

综合来看,多用户可搜索加密已经在授权机制与工程可用性方面取得进展,但在多关键词组合查询与可验证性要求同时存在的情况下,索引维护、授权一致性与验证信息更新将面临更高复杂度。

\subsection{可验证可搜索加密技术研究现状}
在云环境中,服务器出于资源节省、故障或恶意目的可能返回不完整甚至错误的搜索结果。为提升服务可信性,可验证可搜索加密(Verifiable Searchable Encryption, VSE)通过引入认证数据结构或可验证计算技术,使用户在不获取明文索引的前提下验证结果的完整性与正确性\cite{baek2008public}。

现有 VSE 方案常借助哈希链、默克尔哈希树、累加器等工具,为倒排索引或搜索路径构造认证信息,使服务器在返回检索结果的同时给出可验证证明。该思路在单关键词或简单查询条件下较易实现,但在多关键词组合查询中,验证对象往往从“单个倒排列表”扩展为多个列表之间的组合关系,使得证明结构与更新维护成本显著上升。进一步地,在多用户场景下,授权机制会引入额外的资格检验或访问控制流程,验证机制需要与授权一致性相匹配,否则可能出现“授权通过但证明不可验证”或“证明可验证但权限越界”等问题。现有研究在多用户、多关键词与动态更新同时成立的综合设置下仍缺乏兼顾效率与可验证性的统一框架。

综上,面向本文关注的多用户多关键词 SSE 场景,仍存在三方面挑战。首先,多关键词组合检索带来索引与泄露的耦合增长,验证信息的规模控制与更新维护成为瓶颈。其次,多用户授权引入资格检验与撤销管理,系统需要保证授权结果与验证结果的一致性,并防止服务器在授权阶段进行选择性作弊。最后,在保持原有 SSE 检索复杂度优势的前提下引入可验证机制,需要避免额外索引结构导致的存储与通信成本失控,这对可部署性提出了更高要求。

\section{研究目标与内容}
本文面向多用户多关键词对称可搜索加密场景,研究在泄露受限隐私保护模型下引入轻量可验证机制的问题。目标是在保持原有检索效率与动态更新能力的基础上,使授权用户能够验证云服务器在搜索流程中的关键步骤是否按协议执行,并检测搜索结果的完整性与正确性,从而提升密文检索服务的可信性。

围绕上述目标,本文首先给出多用户多关键词 SSE 的系统模型与威胁模型,明确云服务器可能偏离协议的攻击面与验证需求。在此基础上,本文将可验证性需求拆分为两个与工程实现紧密相关的环节:资格检验阶段与搜索结果返回阶段。前者关注服务器是否正确执行授权过滤与成员性判断,后者关注服务器是否返回了与查询条件一致且未被删减的结果集合。

针对资格检验阶段,本文研究如何在现有基于概率数据结构的设计上引入确定性证明,使用户能够验证服务器的过滤判断。针对搜索结果返回阶段,本文研究如何利用更新阶段的可信计算,将“正确结果的校验信息”与既有索引结构融合,在不显著增加索引规模的条件下支持结果正确性验证。最后,本文对提出机制进行安全性与可验证性分析,并通过实验评估验证机制引入后的系统开销与性能影响。

本文的创新点主要体现在两个方面。

第一,提出基于默克尔哈希树的可验证资格检验机制。针对多用户场景中常见的资格检验流程,本文以认证数据结构替换概率性成员判断结构,为服务器的资格检验结果提供确定性的成员性证明与可验证路径,使授权用户能够检测服务器在资格检验阶段的选择性遗漏与伪造行为,从结构层面增强系统在多用户环境下的可验证性。

第二,提出基于嵌入式承诺的搜索结果正确性验证机制。针对服务器在检索阶段可能伪造地址或删减结果的问题,本文在数据更新阶段引入承诺思想,将与“正确搜索结果集合”相关的校验信息嵌入既有索引结构中,使用户在搜索阶段能够对服务器返回的结果集合进行一致性验证。在该机制中,验证信息与索引条目同构维护,无需额外引入独立验证索引,从而在可验证性与系统开销之间取得更适合部署的平衡。

\section{论文结构}
本文共分为五章,各章内容安排如下。

第一章为绪论,阐述研究背景与意义,梳理国内外研究现状并归纳关键挑战,概述本文研究目标、研究内容与创新点,并给出全文结构安排。

第二章介绍可搜索加密基础方案与相关密码学预备知识。首先给出系统模型与威胁模型,其次介绍多用户多关键词 SSE 的典型基础方案与搜索流程,并分析现有资格检验结构及其局限,最后说明后续章节所需的密码学原语与技术。

第三章提出基于默克尔哈希树的可验证资格检验机制。该章围绕资格检验阶段的攻击面与验证需求给出结构设计与算法流程,并对可验证性与安全性进行分析。

第四章提出基于嵌入式承诺的搜索结果正确性验证机制。该章给出服务器伪造或删减结果的攻击模型,描述承诺嵌入与搜索验证流程,并对结果正确性与完整性进行安全性分析。

第五章给出实验评估与总结展望。该章通过实验分析可验证机制引入后的性能与开销变化,验证方案的可行性,并总结全文工作与未来研究方向。
