%%% mode: latex
%%% TeX-master: t
%%% End:

\chapter{绪论}
\section{研究背景与意义}
随着信息技术的飞速发展和数字化转型的深入推进,全球已全面步入大数据时代。根据国际数据公司(IDC)的统计,2020 年全球数据总量约为 64.2 ZB\cite{IDC2020DataAge}。随着云计算、物联网以及人工智能等技术的快速发展,全球数据规模持续高速增长。综合多家市场研究机构的估计结果,截至 2026 年前后,全球数据总量已增长至约 180 ZB的量级。云计算作为数字经济的关键基础设施,凭借其强大的计算能力、海量的存储空间以及按需付费的服务模式,彻底改变了传统的本地数据管理范式。越来越多的政府机构、企事业单位及个人用户选择将本地数据迁移至云端,以实现数据的弹性存储、资源共享与按需访问\cite{armbrust2010view}。

然而,数据外包存储在带来便利的同时,也引发了前所未有的安全与隐私挑战。外包数据往往包含高度敏感的信息,例如电子医疗系统中的个人健康记录、金融领域的交易流水、企业的核心知识产权以及政府的机密档案等。一旦这些数据在云端发生泄露或被非法滥用,将造成不可估量的经济损失和社会影响\cite{ren2012security}。近年来频繁发生的数据泄露事件,如2019年Capital One银行数据泄露案涉及1.06亿用户信息,以及2021年Facebook数据泄露事件影响5.33亿用户,进一步凸显了云环境下数据安全的紧迫性\cite{CNNBiz2019CapitalOneBreach, BusinessInsider2021FacebookBreach}。

在现有的云存储服务架构中,云服务提供商(Cloud Service Provider, CSP)通常被建模为"诚实但好奇"(honest-but-curious)的半可信实体\cite{rivest1978data}。这意味着CSP虽然会忠实地执行用户发起的存储、检索等协议流程,不会主动篡改或破坏数据,但它们出于商业利益、法律强制或好奇心驱动,极有可能利用其掌握的底层基础设施权限,通过分析存储的数据副本、监控用户的查询请求频率以及挖掘数据的访问模式等手段,试图推断出数据所有者的隐私信息或商业机密\cite{yang2006privacy}。此外,外部高级持续性威胁(Advanced Persistent Threat, APT)攻击和内部恶意人员窃密等威胁因素,也使得云环境下的数据安全形势日益严峻\cite{cloud2011nist}。

为应对上述隐私威胁,同时满足《中华人民共和国数据安全法》、《个人信息保护法》以及欧盟《通用数据保护条例》(GDPR)等日益严格的数据合规性要求\cite{PRC2021DataSecurityLaw,GDPR2018EU},数据加密技术被广泛认为是云存储安全的最后一道防线。最直接的方法是在数据上传云端之前,使用AES、RSA等强加密算法对数据进行密文处理。然而,传统的加密技术在保护数据机密性的同时,彻底破坏了数据的自然语义和结构特征,导致云服务器无法在密文状态下执行关键词检索、排序、聚合等常规数据操作\cite{shamir1979share}。在这种模式下,若用户需要查询包含特定关键词的文档,必须先将云端的海量密文数据全部下载至本地,解密后再进行检索。这种"下载-解密-检索"的处理模式不仅造成了巨大的网络带宽浪费和客户端计算资源消耗,在移动终端、物联网设备等计算能力受限的应用场景下更是完全不可行的\cite{boneh2001identity}。

如何在不泄露明文内容的前提下,实现对密文数据的高效检索,成为了学术界和工业界亟待解决的核心矛盾。可搜索加密(Searchable Encryption, SE)技术应运而生,它通过构造特殊的加密索引和搜索陷门(trapdoor),允许服务器在零知识的前提下完成关键词匹配,从而在保护数据隐私和保持数据可用性之间找到了理想的平衡点\cite{song2000practical}。

可搜索加密技术的研究可以追溯到2000年Song等人在IEEE S\&P会议上发表的开创性工作~\cite{song2000practical},该工作首次提出了在加密文档上进行关键词搜索的概念。随后,Goh在其博士论文中进一步完善了可搜索加密的理论基础~\cite{goh2003secure}。2004年,Boneh等人在EUROCRYPT会议上提出了公钥可搜索加密(Public-key Encryption with Keyword Search, PEKS)的概念~\cite{boneh2004public},使得第三方能够在不获知私钥的情况下进行搜索授权。2006年,Curtmola等人在ACM CCS会议上提出了适应性安全的对称可搜索加密方案,建立了更为严格的安全模型,并引入了前向隐私和后向隐私的概念~\cite{curtmola2006searchable}。

近年来,可搜索加密技术在多个维度取得了重要进展。在功能性方面,研究人员提出了支持范围查询、模糊关键词匹配、正则表达式搜索等高级检索功能的方案 \cite{li2010fuzzy,cao2013privacy}。在性能优化方面,基于倒排索引、布隆过滤器、乱序存储等技术的高效方案相继涌现 \cite{kamara2010cryptographic,abdalla2005searchable}。在安全性提升方面,针对文件注入攻击、查询恢复攻击等新型威胁模型,研究者们提出了具有更强隐私保护能力的防御机制 \cite{cash2015leakage,zhang2016all,damie2021highly}。

然而,现有的可搜索加密方案在面向实际应用部署时,仍然存在诸多局限性。首先,大多数方案仅支持单关键词搜索,无法满足用户复杂查询需求。其次,现有方案普遍假设单用户场景,缺乏对多用户权限管理和数据共享的有效支持。最后,现有方案缺乏对搜索结果完整性和正确性的可验证机制,存在恶意服务器返回不完整或错误搜索结果的安全风险 \cite{baek2008public}。

在实际的云存储应用场景中,数据共享与协作访问已成为常态。企业内部的项目文档需要在多个部门间共享,医疗机构的患者记录需要在不同科室间流转,学术研究机构的数据集需要向合作伙伴开放访问权限。这种多用户数据共享模式对可搜索加密技术提出了更高的要求。传统的单用户可搜索加密方案假设只有数据所有者能够生成搜索陷门,这在多用户环境下显然是不现实的。如何在保护数据隐私的同时,实现细粒度的访问控制和权限管理,成为亟待解决的关键问题 \cite{fu2015achieving}。

与此同时,单关键词搜索的局限性在大数据时代愈发凸显。现实中的查询需求往往是复杂和多样化的,用户希望能够通过多个关键词的组合(如合取、析取查询)来精确定位所需信息,或者通过范围查询、排序检索等高级功能来获得更好的搜索体验 \cite{sun2013privacy}。例如,在电子病历系统中,医生可能需要同时搜索"糖尿病"和"高血压"两个关键词来查找具有特定并发症的患者记录;在企业文档管理系统中,员工可能需要搜索包含"年度报告"或"财务分析"等关键词的文档。现有的单关键词可搜索加密方案无法直接支持这类复杂查询,通常需要进行多轮交互或客户端后处理,严重影响了搜索效率和用户体验 \cite{cao2013privacy}。

多关键词可搜索加密在技术实现上面临诸多挑战。首先是索引构造的复杂性,需要设计能够同时支持多个关键词匹配的加密索引结构,同时保持较低的存储开销和检索复杂度。其次是搜索陷门的生成机制,需要考虑不同关键词之间的逻辑关系(合取、析取等)以及查询的可扩展性。最后是隐私保护的增强,多关键词搜索可能泄露更多的访问模式信息,需要设计更为精密的隐私保护机制。

搜索结果的完整性和正确性验证是可搜索加密系统面临的另一个关键挑战。在传统的明文搜索环境中,用户能够直接验证搜索结果的正确性,但在密文搜索环境下,这一问题变得更为复杂。半可信的云服务器可能出于降低计算成本、存储空间不足或恶意目的,有选择地返回部分搜索结果,或者返回与查询条件不匹配的错误结果。这种搜索结果的不完整性或错误性可能导致严重的业务后果,例如医疗诊断中遗漏关键病历信息、金融审计中丢失重要交易记录等。

可验证计算(Verifiable Computation)理论为解决这一问题提供了重要的理论基础。通过将可验证计算技术与可搜索加密相结合,可以构造出支持搜索结果完整性和正确性验证的可验证可搜索加密方案。这类方案不仅要保证搜索过程的隐私性,还要提供高效的验证机制,使得用户能够以较小的计算开销验证服务器返回的搜索结果是否完整且正确。

当前的可验证可搜索加密方案主要基于哈希链、默克尔树、累加器等密码学工具来构造验证信息。然而,现有方案在支持复杂查询(如多关键词搜索)时,验证信息的构造和维护变得异常复杂,验证开销也随之显著增加。如何在保持高效性的同时,为多用户多关键词可搜索加密提供可验证性保障,仍然是一个充满挑战的开放性问题。

综上所述,面向多用户的可验证多关键字可搜索加密方案研究具有重要的理论意义和实用价值。从理论层面,该研究将推动可搜索加密、访问控制、可验证计算等多个密码学分支的交叉融合,丰富和完善相关领域的理论体系。从实践层面,该研究将为云存储服务、大数据处理平台、物联网数据管理等实际应用提供更为安全、高效和可靠的隐私保护解决方案,对推动数字经济的健康发展具有重要意义。
\section{国内外研究现状}
可搜索加密(Searchable Encryption, SE)旨在解决云计算与数据外包环境下加密数据的可用性问题,使得数据拥有者在不泄露明文内容的前提下,能够对密文数据执行关键词搜索操作。其中,对称可搜索加密(Searchable Symmetric Encryption, SSE)由于其较高的效率与较低的系统开销,成为当前研究的主流方向。围绕功能性扩展与安全性增强,SSE 技术经历了从静态到动态、从单关键字到多关键字、从基础泄露模型到前向与后向隐私保护的持续演进。

\subsection{可搜索加密及其安全性}
在早期研究中,Song 等人提出了首个具有实用意义的 SSE 方案,利用流密码对关键词进行加密,并通过陷门机制实现单关键字搜索,从而证明了在密文域中执行搜索操作的可行性。然而,该方案仅适用于静态数据集,且未对访问模式与结果规模泄露进行形式化分析,其安全性主要基于经验假设。随后,Goh 提出的 Secure Indexes 方案引入基于布隆过滤器的安全索引结构,在提升查询效率的同时,首次系统性地讨论了索引安全性问题,但不可避免地引入了假阳性率,并仍然局限于静态数据场景。

Curtmola 等人进一步推动了 SSE 的理论化发展,提出了非自适应与自适应攻击模型下的安全定义,并构造了可证明安全的静态 SSE 方案,为后续研究奠定了形式化安全分析的基础。然而,此类方案在数据更新时需重建索引,难以满足实际应用中频繁更新的需求。为提升 SSE 的通用性,Chase 与 Kamara 提出了结构化加密(Structured Encryption)框架,将可搜索加密推广至更复杂的数据结构,并初步讨论了多用户扩展的可能性,但其计算与存储开销较大,且不支持动态更新。

随着云存储应用对数据动态性的需求不断增强,研究重点逐渐转向动态可搜索加密(Dynamic SSE)。Kamara 等人首次提出支持文档添加与删除操作的动态 SSE 方案,标志着 SSE 从静态模型向动态模型的重要转变。该方案在功能上取得突破,但仍存在更新模式泄露与缺乏前向安全性的问题。随后,Stefanov 等人通过系统级优化,提出了具有较小泄露的实用动态 SSE 方案,在性能与安全性之间取得一定平衡,但依然无法抵御历史查询与更新之间的关联泄露。为进一步提升并发环境下的搜索效率,Kamara 与 Moataz 研究了支持并行搜索的动态 SSE 构造,但其安全模型仍未涵盖前向与后向隐私。

为应对动态 SSE 中由更新操作引发的隐私泄露问题,学术界提出了前向安全(Forward Privacy)与后向安全(Backward Privacy)的概念。Bost 提出的 Σoφos 方案首次实现了前向安全的动态 SSE,能够防止新添加文档泄露与历史查询之间的关联关系。随后,Bost、Minaud 与 Ohrimenko 构造了同时满足前向与后向安全的 SSE 方案,在理论上显著增强了动态搜索的隐私保护能力,但其更新与查询效率相对较低。近年来,研究者进一步将前后向隐私扩展至连接查询、范围查询等更复杂的搜索类型,但相应方案普遍存在构造复杂、泄露模型受限或计算开销较大的问题。

在搜索功能方面,早期 SSE 方案多仅支持单关键字查询,难以满足现实场景中复杂的信息检索需求。Cash 等人在 CRYPTO 2013 提出的方案首次系统性地研究了支持布尔查询的 SSE 构造,能够在静态数据集上实现多关键字查询,并在可扩展性方面取得显著进展。该工作通过优化索引结构与查询流程,使搜索复杂度与匹配文档数量相关,为多关键词 SSE 的工程化应用奠定了基础。然而,该方案仍然局限于静态数据环境,且在查询过程中会泄露一定程度的查询结构信息与访问模式。随着研究的深入,学术界逐渐认识到,多关键词搜索不仅是功能扩展问题,同时也是泄露模型急剧复杂化的问题。在多关键词合取查询中,服务器能够观察到不同关键词对应结果集合的交集关系,从而可能推断关键词之间的相关性或查询语义。为缓解该问题,部分研究尝试通过隐藏结果模式(Result Pattern Hiding)来降低多关键词查询所带来的额外泄露,但相应方案通常需要引入更复杂的加密结构或增加查询交互次数,从而在效率与安全性之间形成新的权衡。在动态可搜索加密背景下,多关键词搜索面临更为严峻的挑战。一方面,动态更新操作会引入更新模式泄露;另一方面,多关键词查询本身又可能放大前后查询之间的关联性。部分研究工作尝试将前向与后向安全机制引入多关键词 SSE 中,通过对索引状态进行周期性演化或采用受限加密原语,降低新旧文档与历史查询之间的关联风险。然而,此类方案通常在更新或查询阶段引入对数级甚至线性级的额外开销,限制了其在大规模数据场景下的实用性。

总体来看,多关键词可搜索加密在功能性和效率方面已取得显著进展,但现有方案大多仍局限于单用户模型,且缺乏对搜索结果正确性和完整性的验证能力。
\subsection{多用户可搜索加密研究现状}
在云端数据外包与共享协作成为常态的背景下,可搜索加密系统往往需要支持“一个数据拥有者(Data Owner)将加密数据共享给多个授权用户(Data User)”的多用户(multi-user/multi-client)使用模式。与单用户场景相比,多用户场景不仅要求服务器能够对密文索引执行检索,还要求系统具备细粒度授权、用户动态加入与撤销、以及对抗用户串通(collusion)的能力;否则,一旦某个授权用户密钥泄露或与服务器合谋,就可能危及整个数据集的机密性与查询隐私。

最直接的扩展方式是由数据拥有者与所有用户共享同一把搜索密钥,使所有授权用户均可生成陷门并发起搜索。然而,该方式在安全性和可管理性上都存在明显缺陷:其一,任意单个用户被攻破都会导致系统级密钥暴露;其二,难以实现高效的用户撤销与权限变更;其三,在多用户环境下更容易出现“跨用户泄露”(cross-user leakage)与串通攻击风险。因此,近年的研究逐渐转向在更严格威胁模型下构建多用户可搜索加密,并围绕“授权机制如何实现”形成了几条典型技术路线。

(一)多客户端/多用户对称可搜索加密(Multi-client SSE)路线。该类工作通常以高效 SSE 架构为基础,通过引入用户侧授权密钥、关键词能力限制(capability)、或额外的授权辅助信息,实现“数据拥有者授权多个客户端执行受限搜索”的目标。一些研究明确将多客户端布尔查询与动态更新纳入同一框架,使数据拥有者能够限制不同用户的可搜索关键词集合,并支持非交互式或低交互式查询授权,从而提升实用性与可部署性。例如,已有工作提出动态多客户端 SSE 框架以支持授权布尔查询并在动态数据库上工作,强调在多客户端条件下对授权粒度与效率的平衡。 也有研究进一步关注“前向隐私”(forward privacy)等更强安全属性在多客户端场景下的实现,给出了面向云存储访问控制的更系统化方案。 近年还有面向多客户端与多关键字同时成立的对称方案探索,反映出研究趋势正在从“能共享”走向“可扩展、可表达”。

(二)基于属性的访问控制与可搜索结合(ABE/ABKS)路线。属性基加密(ABE)可将访问策略直接嵌入密文或密钥,实现细粒度授权与天然的抗串通能力,被广泛视为多用户数据共享的核心密码学工具之一。ABE 的思想可追溯到 Sahai 与 Waters 提出的“模糊身份/属性”加密范式,强调基于属性集合的解密能力与抗串通特性。 随后 Bethencourt 等提出 CP-ABE(Ciphertext-Policy ABE),使数据拥有者能用访问结构表达策略,从工程角度推动了 ABE 在访问控制场景的应用。 在此基础上,研究者进一步将 ABE 与关键词检索结合,形成属性基关键词搜索(ABKS)及其增强形式,使“只有满足策略的用户才能生成有效搜索令牌或解密搜索结果”成为可能,并针对关键词猜测攻击等威胁给出更严格的安全模型与构造。 这一路线的优势在于授权语义清晰、便于表达复杂权限关系;不足在于通常带来较高的配对运算/索引开销,且在动态撤销与高频更新时容易出现性能瓶颈。

(三)代理重加密与广播授权(PRE/BPRE)路线。代理重加密(Proxy Re-Encryption, PRE)允许半可信代理在不获知明文的情况下,将“对 A 加密的密文”转换为“对 B 加密的密文”,为云端数据共享与权限迁移提供了强有力的密码学原语。该概念最早由 Blaze 等提出,随后 Ateniese 等给出了更系统的方案族并推动其在安全分布式存储中的应用。 在云共享语境下,PRE 常被用于实现用户加入/撤销、跨域共享与最小化数据拥有者在线负担;进一步地,广播代理重加密(BPRE)等工作尝试将“对一组用户授权”的能力与动态用户管理结合,以降低频繁重加密与密钥分发成本。 该路线通常更利于工程落地与权限迁移,但需要谨慎处理代理/服务器的能力边界,以及重加密密钥泄露、撤销一致性等安全细节。

\subsection{可验证多关键词可搜索加密技术研究现状}
在云计算环境下,云服务器通常被建模为半可信实体,其可能返回不完整或错误的搜索结果。因此,仅保证搜索过程的隐私性已无法满足实际应用需求,可验证可搜索加密(Verifiable Searchable Encryption,VSE)逐渐受到研究者关注。

早期的可验证可搜索加密方案主要针对单关键词搜索场景,通常借助哈希链或 Merkle 哈希树构造验证结构,使用户能够验证搜索结果的完整性和正确性。例如,部分研究通过为倒排索引引入认证信息,使服务器在返回搜索结果的同时提供相应的验证证明。然而,这类方案在多关键词搜索场景下难以直接扩展,其验证信息规模和计算开销随关键词数量显著增加。

为提升可验证搜索的表达能力,有研究尝试将累加器、可验证计算等技术引入可搜索加密体系,实现更紧凑的验证证明。但在多关键词、多用户环境中,如何在保证隐私性的同时构造高效、可扩展的验证机制,仍然面临诸多挑战。一方面,多关键词搜索需要验证不同关键词条件的组合关系;另一方面,多用户数据共享场景下还需考虑不同用户权限对应的验证一致性问题。

国内关于可验证可搜索加密的研究仍处于探索阶段,现有工作多集中于单关键词或静态数据集场景,对于动态更新、多用户授权以及复杂查询条件的综合支持仍显不足。

\section{研究目标与内容}
综合前述研究背景与国内外研究现状可以看出,现有可搜索加密方案在功能性和安全性方面虽已取得大量研究成果,但在实际云存储应用中,仍然缺乏一种能够同时支持多用户访问、多关键字查询以及搜索结果可验证性的统一解决方案。

近年来提出的多客户端动态可搜索加密方案,尤其是基于令牌化授权机制的研究工作,在支持多用户查询、动态数据更新以及前向隐私与后向隐私等方面表现出较好的实用性。然而,这类方案通常假设云服务器在搜索过程中能够诚实执行协议,缺乏对搜索结果完整性与正确性的验证机制。当服务器出于降低计算成本或恶意行为而返回不完整或错误结果时,授权用户难以有效发现并加以防范。

另一方面,可验证可搜索加密相关研究为解决云服务器不可信问题提供了重要思路,但现有方案多集中于单用户或单关键词搜索场景,在多用户授权、多关键字组合查询以及动态更新并存的复杂环境下难以直接适用。特别是在引入前向隐私和后向隐私等更强安全属性后,验证机制的设计与维护将面临更高的复杂性。

基于上述分析,本文的研究目标是:在现有多客户端动态可搜索加密方案的基础上,引入可验证搜索机制,构建一种支持多用户访问、多关键字查询且具备搜索结果可验证性的可搜索加密方案。该方案旨在在保持原有方案高效性与强隐私保护特性的前提下,使授权用户能够有效验证云服务器返回搜索结果的完整性与正确性,从而提升云端密文检索服务的整体可信性

\subsection{研究内容}
在多用户多关键词可搜索加密场景下,云服务器在搜索过程中可能偏离协议执行,导致搜索结果不完整或不正确,而现有方案在结果可信性与可验证性方面仍存在不足。针对上述问题,本文围绕多用户多关键词 SSE 的系统模型与搜索流程,对可验证搜索机制展开研究,重点关注资格检验阶段与搜索结果返回阶段的可验证性问题。

本文的主要研究内容包括以下几个方面。

首先,在分析多用户多关键词 SSE 系统模型与威胁模型的基础上,系统梳理现有方案在资格检验与搜索结果返回阶段可能存在的安全隐患,明确服务器不诚实行为对搜索正确性与完整性造成的影响,并据此给出本文的研究问题定义与设计目标。

其次,针对现有方案中资格检验阶段可验证能力不足的问题,本文从索引与数据结构层面入手,设计了一种基于 Merkle Hash Tree 的可验证资格检验机制。通过引入确定性的成员性证明手段,使用户能够对服务器在资格检验阶段的执行结果进行验证,从而提升系统在多用户场景下的可验证性与安全性。

再次,针对服务器在搜索阶段可能伪造或篡改搜索结果的问题,本文进一步研究搜索结果返回阶段的正确性验证机制。通过在数据更新阶段引入嵌入式承诺思想,将搜索结果相关的校验信息与现有索引结构相结合,使用户能够在搜索阶段对服务器返回结果的正确性进行有效验证。

在上述研究工作的基础上,本文对所提出机制的安全性与可验证性进行了分析,并通过实验评估对相关机制在性能与开销方面的影响进行了验证,以说明所引入可验证机制在保证安全性的同时,对系统效率的影响保持在可接受范围内。
\subsection{研究创新点}
综上所述,本文的主要创新点可以概括为以下两个方面。

一是提出了一种基于 Merkle Hash Tree 的可验证资格检验机制,从结构层面增强了多用户多关键词 SSE 方案在资格检验阶段的可验证能力,为防止服务器在资格检验过程中的不当行为提供了有效手段。

二是设计了一种基于嵌入式承诺的搜索结果正确性验证机制,在不额外引入独立索引结构的前提下,实现了对服务器返回搜索结果正确性与完整性的验证,拓展了多用户多关键词 SSE 场景下的可验证搜索研究思路。
% \clearpage

% \clearpage
\subsection{论文结构}
本文共分为五个章节,整体结构围绕多用户多关键词可搜索加密场景下的可验证搜索问题展开,各章节内容安排如下。

第一章为绪论,主要介绍本文的研究背景与研究意义,分析多用户多关键词 SSE 场景下面临的可验证性问题,在此基础上概述本文的研究内容与创新点,并对全文结构安排进行说明。

第二章介绍可搜索加密的基础方案与相关密码学预备知识。首先给出多用户多关键词 SSE 的系统模型与威胁模型,随后介绍典型基础方案及其搜索流程,分析现有方案在资格检验阶段的设计方式及其局限性,最后对本文后续研究中涉及的密码学原语与基本技术进行说明,为后续章节的机制设计与安全分析提供理论基础。

第三章针对资格检验阶段可验证性不足的问题,提出一种基于 Merkle Hash Tree 的可验证资格检验机制。本章首先分析资格检验阶段可能存在的安全风险与设计需求,随后给出具体结构设计与算法流程,并对所提出机制的可验证性与安全性进行分析。

第四章进一步关注搜索结果返回阶段的正确性问题,提出一种基于嵌入式承诺的搜索结果正确性验证机制。本章通过分析服务器伪造或篡改搜索结果的攻击模型,给出相应的机制设计与搜索验证流程,并对搜索结果正确性与完整性进行安全性分析。

第五章对本文所提出的机制进行实验评估与总结。通过实验分析所引入可验证机制在性能与系统开销方面的影响,验证其在保证安全性的同时对系统效率的影响情况,并在此基础上总结全文工作,展望未来的研究方向。

% 第六章 总结与展望:本章对全文的研究工作进行总结,指出本文研究的不足和未来的展望。

% % \section{课题来源}
% % 本课题来源于:

% % 国家自然科学基金面上项目:基于多输出支持向量回归的预测技术研究(立项时间:2016-01,项目编号:71571080)。

% % 国家自然科学基金面上项目:大数据环境下基于计算智能的预测建模技术及其在电力负荷预测中的应用(立项时间:2019-01,项目编号:71871101)。

% % 国家电网公司华中分部科技项目:华中区域共享型电力交易与服务平台关键技术研究(立项时间:2020-03,项目编号:52140019000U)。